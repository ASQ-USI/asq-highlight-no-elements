%This LaTeX Template was designed for use freely by anyone for student work and research
%within the Università della Svizzera Italiana and cannot be sold.
%Copyright © Jacques DAFFLON -- jacques.dafflon@gmail.com - 2011

\documentclass[a4paper,10pt]{article}
\usepackage[utf8x]{inputenc}
\usepackage[top=2.5cm, bottom=3cm, left=2cm, right=2cm]{geometry} %Redfining the margins
\usepackage{titling} %for using title data such as '\theauthor'
\usepackage{fancyhdr} %Allows to customize header and footnotes
\usepackage{hyperref} %Creates links and bookmarks on the pdf file
\usepackage{helvet} %Using Helvetica Font
\renewcommand{\familydefault}{\sfdefault} %Sans Serif as default
\usepackage{amsmath}
\usepackage{amssymb}


% Title Page
\title{Different Editor plugins review} %DON'T FORGET TO FILL IN!
\author{Margarita \textsc{Grinvald} } 
\date{July 26, 2013} %CHECK THE DATE!

%Variables
\def\school{\textsc{Università della Svizzera Italiana}} %School name
\def\faculty{\textsc{Faculty of Informatics}} %Faculty name
\def\course{Summer 2013 UROP Internship: InformaWeb} %DON'T FORGET TO FILL IN!

%Header and footer
%Do not edit here, only above!
\fancyhf{} %reset everything
\pagestyle{fancy}
\fancypagestyle{plain}{}
\setlength{\headheight}{12pt}
\lhead{\theauthor}
\rhead{\thedate}
\lfoot{\footnotesize{\school\\\faculty}}
\cfoot{\small page \thepage}
\rfoot{\footnotesize{\course\\\thetitle}}


\begin{document}
\maketitle
%Start writting here.

In InformaWeb we try to implement a new type of question, based on highlighting text.\\
The highlight will work on top of a code editor with Syntax Highlight feature.\\\\
Typical usage scenarios are:
\begin{enumerate}
\item Teacher providing correct solutions to the questions by highlighting pieces of text
\item Students highlighting keywords
\item Students highlighting blocks representing different structures to identify 
\end{enumerate}


\section*{User Experience Requirements}

\begin{enumerate}
\item Highlighting adhers to existing standards (will behave like in well known text-editors such as Word, with different color options.)
\item Availabilty of arbitrary \emph{n} visually distinct colors for different types of elements to identify in the piece of code.
\item Dehighlighting as intuitive as possible, either by allowing the user to remove only entire blocks of highlighted text or by allowing the inverse of the highlighting process.

\end{enumerate}

\section*{Implementation Details}

\subsection*{Events fired}

On user's cursor activity an event is fired and a check needs to be done, whether it was just a \emph{mouseClick} or some piece of text was selected.

\paragraph*{CodeMirror}
The event that fires is \emph{"cursorActivity"}, and with the help of \emph{editor.somethingSelected()} method the check is easily and correctly done by the plugin. 

\paragraph*{Ace Editor}
The events available are two: \emph{changeCursor} and \emph{changeSelection}.\\ With the first the method \emph{editSession.selection.isEmpty()} always returns \emph{false}, meaning that a simple click will trigger the creation of a new selection marker from the previous cursor position to the current one. \\
The second event is more "secure", meaning that simple clicks won't create new selections in the text, but the event is fired three times instead of one when a simple click occurs. This causes a problem, since the first time it's fired the method \emph{editSession.selection.isEmpty()} returns \emph{false}, causing therefore the creation of a new selection mark in the text.\\
To avoid the problem I was suggested to call the check method \emph{isEmpty()} synchronously. This helps to solve the issue, and properly use \emph{changeSelection} to properly respond to user activity and if needed add a selection marker.


\end{document}          
